\documentclass[10pt]{article}

\usepackage{fullpage}
\usepackage{color}
\usepackage[table]{xcolor}
\usepackage{hyperref}
\usepackage{graphicx}

%%%%%%%%%%%%%%%%
% Miscellaneous
%%%%%%%%%%%%%%%%

\definecolor{primary}{rgb}{0,0,.50}
%\definecolor{primary}{rgb}{0,0,0}
\definecolor{secondary}{rgb}{.7,.5,0}
\definecolor{table-primary}{rgb}{1,1,1}
\definecolor{table-secondary}{rgb}{.975,.99,1}

%%%%%%%%%%%%%%%%
% Title Section
%%%%%%%%%%%%%%%%

\title{\color{primary}\texttt{System and Unit Test Report \\ UCSC Plaza}}
\author{{\color{secondary}\textbf{Team Amlesh the Great}} \\ Kyungmin So (PO), Youngsoo Jang, \\ Hobin Ryu, Seungwoo Lee \\ Amlesh Sivanantham, and James Garbagnati }
\date{July 25, 2016}

%%%%%%%%%%%%%%%%%%%%%%
% User-Defined Macros
%%%%%%%%%%%%%%%%%%%%%%

%\ignore = Multiline comments
\newcommand{\ignore}[1]{}

\newcommand{\gobblepagenum}{\thispagestyle{empty}\addtocounter{page}{-1}}

%\fancysec{Section title}{Label} = Fancy, colored and labelled section
% Basically just to make it easier to change the format of the whole doc
% Commands with an X are non-numbered sections
\newcommand{\fancysec}[2] {{\color{primary}\section{#1} \label{sec:#2}}}
\newcommand{\fancysub}[2] {{\color{primary}\subsection{#1} \label{sec:#2}}}
\newcommand{\fancysubsub}[2] {{\color{primary}\subsubsection{#1} \label{sec:#2}}}

\newcommand{\fancysecX}[2] {{\color{primary}\section*{#1} \label{sec:#2}}}
\newcommand{\fancysubX}[2] {{\color{primary}\subsection*{#1} \label{sec:#2}}}
\newcommand{\fancysubsubX}[2] {{\color{primary}\subsubsection*{#1} \label{sec:#2}}}


% P.S. I'm so fancy, you already know


%%%%%%%%%%%%%%%%%
% Begin Document
%%%%%%%%%%%%%%%%%
\begin{document}

\maketitle

\fancysec{System Test Scenarios}{systemTestScenarios}
	\fancysub{Sprint 1}{sprint1}
		\begin{itemize}
			\item User Story 1: As a member of the UCSC community, I want to be able to log in to an unique account made for me, so that I can use the application ``UCSC Plaza."
			
			\begin{itemize}
				\item Go to UCSC Plaza Main Page.
				\item Click on the "Not a Member? Sign Up" link, type input details
				\begin{itemize}
					\item username = (cmps115)
					\begin{itemize}
						\item Click the Check Duplication
						\item Get Confirmation that username is not taken.
					\end{itemize}
					\item password = (cmps115pass)
					\item password confirmation = (cmps115pass)
					\item First Name  = (Alexander)
					\item Last Name = (Hamilton)
					\item Major = (Computer Science)
					\item Division = (Jack Basking School of Engineering)
					\item email = (asivanan@ucsc.edu)
				\end{itemize}
				\item Click Verify Email button. Go check your email for code.
				\begin{itemize}
					\item Input the private key
					\item Click the Verify button
					\item A success alert will popup
				\end{itemize}
				\item Press the confirm button
				\item Taken back to the main Page
				\item input username and password just made and click Login
			\end{itemize}
			
			      
			\item User Story 2: As an event planner, I want my event to be shown to members of the UCSC community, so that I can advertise strictly to UCSC members.
			\begin{itemize}
				\item This is shown in the previous system test. Only UCSC email addresses can be verified.
			\end{itemize}
			
			\item User Story 3: As an event planner, I want to add or delete my events, so that other members of UCSC can see if my event is available.
			\begin{itemize}
				\item In the Main Page, click on the Create Event
				\item Redirected to the Create Event page; and Input Details
				\begin{itemize}
					\item Event title = (Acceptance Test)
					\item Marker Location = ...
					\begin{itemize}
						\item Move map and click on the Area closest to BE316.
						\item A coordinate marker will be place.
					\end{itemize}
					\item Detailed Description = (Write anything)
					\item INSERT MAIN PICTURE = ...
					\begin{itemize}
						\item Clicking it opens up a file manager
						\item Open a picture file
					\end{itemize}
					\item Major = (Physics)
					\item Category = (Party)
					\item Starting Time = (2016 Jul 28 18 00)
					\item Ending TIme = (2016 Jul 28 18 30)
					\item Event Location = (BE 316)
					\item Max Attendance = (8)
					\item Homepage = ()
				\end{itemize}
				\item Hit the confirm button to add event.
				\item To delete this event, Look at event list and find your event.
				\item Click on it to open details
				\item Press the delete button.
			\end{itemize}

			\item User Story 4: As an event-goer, I want to see a map of the UCSC campus, so that I know where events are planned to happen.
			\begin{itemize}
				\item Load Main Page
				\item See the google map
			\end{itemize}

			\item User Story 5: As an event-goer, I want to be able to see details of an event, so that I can decide whether I want to attend or not.
			\begin{itemize}
				\item In the Event List click on an Event
				\item View Details of the event
				\item Click on the More Info link
				\item View Detailed View of the event
			\end{itemize}
		\end{itemize}
	\fancysub{Sprint 2}{sprint2}
    
	    \begin{itemize}
	    	\item User Story 1: As an event planner, I want to add or delete my events, so that other members of UCSC can see if my event is available.
	    	\begin{itemize}
	    		\item This System test is already implemented in the previous sprint
	    	\end{itemize}
	    	
	    	\item User Story 2: As an event-goer, I want to be able to see details of an event, so that I can decide whether or not to go.
	    	\begin{itemize}
	    		\item This System test is already implemented in the previous sprint
	    	\end{itemize}
	    	
	    	\item User Story 3: As an event-goer, I want to be able to search for specific events, so that I may find the event that I need.
	    	\begin{itemize}
	    		\item Type Search Query Soccer in the Search field.
	    		\item Select the Search Menu button and input the following
	    		\begin{itemize}
	    			\item Major = (Major)
	    			\item Category = (Sports)
	    			\item Start Time = (2016 Jul 28 18 00)
	    			\item End Time = (2016 Dec 01 18 00)
		    	\end{itemize}
		    	\item Press the Search Icon to begin search
		    	\item View the Event list 
		    \end{itemize}
	  
	    	\item User Story 4: As an event-goer, I want to be able to see a marker on the map, so that I can see where it is.
	    	\begin{itemize}
	    		\item Open Main Page.
	    		\item Look at Google map and identify markers.
	    		\item Click on an event in Event List.
	    		\item Look at Google map and identify marker. related to event. It will be bouncing.
	    	\end{itemize}
	    	
	    	\item User Story 5: As an event-goer, I want to be able to send or rescind an application to a specific event, so that the event planner knows whether I will attend or not.
	    	\begin{itemize}
	    		\item Open Main Page.
	    		\item Click on an Event to open expanded view.
	    		\item Click on the Enter Button to apply.
	    		\item Click on the Leave Button to leave.
	    	\end{itemize}
	
	    	\item User Story 6: As an event planner, I want to be able to see who applied to my event and be able to accept or decline their application, so that event-goers know if they are allowed to participate or not.
	    	\begin{itemize}
	    		\item Go to My Page
	    		\item Click on the Hosted Events Tab
	    		\item Click on your event
	    		\item See List Of users who applied to your event
	    	\end{itemize}
	    
	    	\item User Story 7: As an event planner, I want to be able to manage my event, so that I can apply certain constraints to the event.
	    	\begin{itemize}
	    		\item Go to My Page
	    		\item Click on the hosted Events tab
	    		\item Click on your event
	    		\item See the Waiting tab to see list of users currently waiting to find out status.
	    		\begin{itemize}
	    			\item Press green check-mark button next to user's name to accept them.
	    			\item Press red X-mark button next to user's name to reject them.
		    	\end{itemize}
		    	\item See the Accepted tab to see list of users you have accepted.
		    	\begin{itemize}
		    		\item Press yellow clock button next to user's name to send them back to waiting
		    	\end{itemize}
		    	\item See the Rejected tab to see list of users you have rejected.
		    	\begin{itemize}
		    		\item Press yellow clock button next to user's name to send them back to waiting
		    	\end{itemize}
	    	\end{itemize}
	    	
	    \end{itemize}

	\fancysub{Sprint 3}{sprint3}
	
		\begin{itemize}
			\item User Story 1: As an event planner, I want to add or delete my events, so that other members of UCSC can see if my event is available.
	    	\begin{itemize}
	    		\item This System test is already implemented in the previous sprint
	    	\end{itemize}

			\item User Story 2: As an event-goer, I want to be able to see details of an event, so that I can decide whether or not to go.
	    	\begin{itemize}
	    		\item This System test is already implemented in the previous sprint
	    	\end{itemize}
	    	
			\item User Story 3: As an event-goer, I want to be able to search for specific events, so that I may find the event that I need.
	    	\begin{itemize}
	    		\item This System test is already implemented in the previous sprint
	    	\end{itemize}
	    	
			\item User Story 4: As an event-goer, I want to be able to send or rescind an application to a specific event, so that the event planner knows whether I will attend or not.
	    	\begin{itemize}
	    		\item This System test is already implemented in the previous sprint
	    	\end{itemize}
	    	
			\item User Story 5: As an event planner, I want to be able to see who applied to my event and be able to accept or decline their application, so that event-goers know if they are allowed to participate or not.
	    	\begin{itemize}
	    		\item This System test is already implemented in the previous sprint
	    	\end{itemize}

			\item User Story 6: As an event planner, I want to be able to manage my event, so that I can apply certain constraints to the event.
	    	\begin{itemize}
	    		\item This System test is already implemented in the previous sprint
	    	\end{itemize}
			
			\item User Story 7: As an event-goer, I want to see the current and maximum attendance for the event, so that I can ensure there is space for me to attend and that it is not overcrowded.
			\begin{itemize}
				\item Go to Main Page and find your Event.
				\item Open the expanded view.
				\item Look at the Person Icon to see how many have been accepted out of the total number allowed.
			\end{itemize}
			
			\item User Story 8: As an event-planner, I want to ensure that my event does not exceed maximum attendance, so that my event is not overcrowded and has enough room for all event-goers.
			\begin{itemize}
				\item Go to My Page
				\item Go to the Hosted Events tab
				\item Look at your event but do not click on it.
				\begin{itemize}
					\item The Person Icon shows Max Allowed to this event
					\item The Clock Icon shows how many people are waiting
					\item the Green Checkmark shows how many people have been accepted.
				\end{itemize}
			\end{itemize}
			
		\end{itemize}
	\fancysec{Unit Tests}{unitTest}
		\begin{itemize}
			\item Please see the files alongside this file with in the naming scheme testingPersonName.txt. This contain the hand written unit tests.
		\end{itemize}
	
	
\end{document}
