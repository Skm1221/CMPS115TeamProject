\documentclass[10pt]{article}

\usepackage{fullpage}
\usepackage{color}
\usepackage[table]{xcolor}
\usepackage{hyperref}
\usepackage{graphicx}

%%%%%%%%%%%%%%%%
% Miscellaneous
%%%%%%%%%%%%%%%%

\definecolor{primary}{rgb}{0,0,.50}
%\definecolor{primary}{rgb}{0,0,0}
\definecolor{secondary}{rgb}{.7,.5,0}
\definecolor{table-primary}{rgb}{1,1,1}
\definecolor{table-secondary}{rgb}{.975,.99,1}

%%%%%%%%%%%%%%%%
% Title Section
%%%%%%%%%%%%%%%%

\title{\color{primary}\texttt{UCSC Plaza \\ Sprint 1:}}
\author{{\color{secondary}\textbf{Team Amlesh the Great}} \\ Kyungmin So (PO), Youngsoo Jang, \\ Hobin Ryu, Seungwoo Lee \\ Amlesh Sivanantham, and James Garbagnati }
\date{Release: American Bobtail (July 5, 2016) \\ Revision 0.0 (July 5, 2016)}

%%%%%%%%%%%%%%%%%%%%%%
% User-Defined Macros
%%%%%%%%%%%%%%%%%%%%%%

%\ignore = Multiline comments
\newcommand{\ignore}[1]{}

\newcommand{\gobblepagenum}{\thispagestyle{empty}\addtocounter{page}{-1}}

%\fancysec{Section title}{Label} = Fancy, colored and labelled section
% Basically just to make it easier to change the format of the whole doc
% Commands with an X are non-numbered sections
\newcommand{\fancysec}[2] {{\color{primary}\section{#1} \label{sec:#2}}}
\newcommand{\fancysub}[2] {{\color{primary}\subsection{#1} \label{sec:#2}}}
\newcommand{\fancysubsub}[2] {{\color{primary}\subsubsection{#1} \label{sec:#2}}}

\newcommand{\fancysecX}[2] {{\color{primary}\section*{#1} \label{sec:#2}}}
\newcommand{\fancysubX}[2] {{\color{primary}\subsection*{#1} \label{sec:#2}}}
\newcommand{\fancysubsubX}[2] {{\color{primary}\subsubsection*{#1} \label{sec:#2}}}


% P.S. I'm so fancy, you already know


%%%%%%%%%%%%%%%%%
% Begin Document
%%%%%%%%%%%%%%%%%
\begin{document}

\maketitle

\fancysecX{High Level Goals}{goals}

    \begin{itemize}
        \item Create a basic event manager and advertiser for the UCSC community.
        \item Visualize locations of events using Google maps API.
%        \item Search functionality to find events happening within campus.
%        \item Give users the ability to rate and comment to leave feedback on events.
%        \item Suggest similar events to users based on their personal preferences.
    \end{itemize}
    
\fancysecX{User Stories for Release}{stories}
    
    \fancysubX{Sprint 1}{sprint1}
     
        \begin{itemize}
            \item (5) User Story 1: As a member of the UCSC community, I want to be able to log in to an unique account made for me, so that I can use the application ``UCSC Plaza."

                \begin{itemize}
                    \item Task 1. Create a Login prompt (2 hours)
                    \item Task 2. Create a way for user to Sign Up and be able to choose college affiliation. (2 hours)
                    \item Task 3. Check for ID duplication. (1 hours)
                \end{itemize}        
                Total for User Story: 5 hours
            \item (3) User Story 2: As an event planner, I want my event to be shown to members of the UCSC community, so that I can advertise strictly to UCSC members.
            
                \begin{itemize}
                    \item Task 1. Setup an E-Mail Validation System (Verification). (2 hours)
                \end{itemize}
                Total for User Story: 2 hours
            \item (8) User Story 3: As an event planner, I want to add or delete my events, so that other members of UCSC can see if my event is available.
            
                \begin{itemize}
                    \item Task 1. Allow user to upload an Event. (4 hours)
                    \item Task 2. Allow user to remove an Event. (3 hours)
                \end{itemize}
                Total for User Story: 7 hours
            \item (5) User Story 4: As an event-goer, I want to see a map of the UCSC campus, so that I know where events are planned to happen.
            
                \begin{itemize}
                    \item Task 1. Create an interface for the map of the UCSC campus with events displayed on it. (3 hours)
                \end{itemize}
                Total for User Story: 3 hours
            \item (3) User Story 5: As an event-goer, I want to be able to see details of an event, so that I can decide whether I want to attend or not.
        
                \begin{itemize}
                    \item Task 1. Display the details of an event. (2 hours)
                \end{itemize}
                Total for User Story: 2 hours
        \end{itemize}
     
%    \fancysubX{Sprint 2}{sprint2}
        
%        \begin{itemize}
%            \item (5) User Story 1: As an event planner, I want to be able to manage my event, so that I can apply certain constraints to the event.
%            \item (3) User Story 2: As an event-goer, I want to be able to send or rescind an application to a specific event, so that the event planner knows whether I plan to attend or not.
%            \item (3) User Story 3: As an event planner, I want to see who applied to my event and be able to accept or decline their application, so that the event goers know if they are allowed to partake or not.
%            \item (5) User Story 4: As an event-goer, I may wish for privacy settings, so that I can ensure that my event-attendance is not seen by the public.
%            \item (8) User Story 5: As an event-goer, I want to be able to search for specific events, so that I may apply to attend the event.
%        \end{itemize}
         
%    \fancysubX{Sprint 3}{sprint3}
     
%        \begin{itemize}
%            \item (5) User Story 1: As an event-goer, I want to see the current and maximum attendance for the event, so that I can ensure there is space for me to attend and that it is not overcrowded. 
%            \item (5) User Story 2: As an event-planner, I want to ensure that my event does not exceed maximum attendance, so that my event is not overcrowded and has enough room for all event-goers.
%            \item (3) User Story 3: As an event-goer, I want to be able to see and submit my own rating for an event, so that I can decide to go to an ongoing event or offer my opinion towards the event.
%            \item (3) User Story 4: As an event planner, I want to get some feedback on my event, so that I can improve the experience of my event.
%            \item (8) User Story 5: As an event-goer, I want to be able to socialize over an event, so that I can share my opinion with friends and other participants.
%            \item (21) User Story 6: As an event-goer, I want to be able to see if my friends are going to a certain event, so that I know which events have my friends in it.
%            \item (13) User Story 7: As an event-goer, I want to receive recommendations on other events tailored to my preference, so that I can see events that I may attend in the future.
%        \end{itemize}

\fancysecX{Team Roles}{roles}

    \begin{itemize}
        \item Kyungmin So: Product Owner, Back-end Developer
        \item Youngsoo Jang: Scrum Master, Front-end Developer
        \item Hobin Ryu: Front-end Developer
        \item Seungwoo Lee: Designer
        \item Amlesh Sivanantham: Front-end Developer
        \item James Garbagnati: Front-end Developer
    \end{itemize}

    \fancysubX{Intial Task Asignment}{intialtask}
        \begin{itemize}
            \item Kyungmin So: User Story 2, Task 1
            \item Youngsoo Jang: User Story 1, Task 1
            \item Hobin Ryu: User Story 4, Task 1
            \item Seungwoo Lee: Designer
            \item Amlesh Sivanantham: User Story 5, Task 1 
            \item James Garbagnati: User Story 3, Task 1
        \end{itemize}

\fancysecX{Initial Burnup Chart}{burnupchart}

    Intial Burnup Chart not available at this time.

\fancysecX{Initial Scrum Board}{scrumboard}

   Intial Scrum Board not available at this time.

\fancysecX{Scrum Times}{scrumTimes}

    We will be meeting on Wednesday, Thursday, and Friday at 4:30 PM

\end{document}
